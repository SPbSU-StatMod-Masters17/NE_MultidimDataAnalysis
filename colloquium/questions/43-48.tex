\subsection{Билет 43. Общность как множественный коэффициент корреляции.}
Тут вроде даже сама НЭ не может ответить что надо говорить. А я уж тем более...

%%%%%%%%%%%%%%%%%%%%%%%%%%%%%%%%%%%%%%%%%%%%%%%%%%%%%%%%%%%%%%%%%%%%%%%%%%%%%%%%%%%%%%%%%%%
\subsection{Билет 44. Как интерпретируются признаки в ФА?}


\begin{gather*}
    \mathbb{F} = 
    \begin{pmatrix}
        f_{11} &\ldots & f_{1r}\\
        \vdots & \vdots & \vdots\\
        f_{p1}& \ldots & f_{pr}\
    \end{pmatrix}, f_{ij} = \rho(\xi_i, \eta_j)
\end{gather*}
Специфика факторного анализа в том, что факторные значения нам вообще не нужны, в базовой постановке (для интерпретации). Для их интерпретации хватает и $\mathbb{F}$. То есть можно сказать, что элементы матрицы факторных нагрузок --- это корреляция между исходными признаками и факторными значениями. С чем коррелирует фактор, то и объясняет. 


%%%%%%%%%%%%%%%%%%%%%%%%%%%%%%%%%%%%%%%%%%%%%%%%%%%%%%%%%%%%%%%%%%%%%%%%%%%%%%%%%%%%%%%%%%%
\subsection{Билет 45. Зачем нужны вращения в ФА? Как устроены ортогональные вращения?}
Предположим, что мы нашли $\tilde\xi = \mathbb{F}\eta$. Пусть $\mathbb{W}$ --- ортогональна матрица вращения в $R^{r}$, $\eta' = \mathbb{W}\eta$. При этом после ортогонального вращения факторных значений они так и останутся ортогональными (если были исходно ортогональными). Тогда 
\begin{equation}
\tilde\xi = \mathbb{F}\mathbb{W}\mathbb{W^T}\eta \footnote{Ясно, что это вектор} = \mathbb{F'}\eta.
\end{equation}
  Модель при этом остается верной. То есть факторы определяются не единственным образом. При этом $\eta'$ удовлетворяет тем же условиям. Выбирая матрицу вращения, мы можем упрощатиь интерпретацию факторов. Если мы возьмем $r = p$ при вращении, то мы получим тоже самое (не корректный результат). 

%%%%%%%%%%%%%%%%%%%%%%%%%%%%%%%%%%%%%%%%%%%%%%%%%%%%%%%%%%%%%%%%%%%%%%%%%%%%%%%%%%%%%%%%%%%
\subsection{Билет 46. Вращение по методу varimax.}
\paragraph{Интерпретация факторов в ФА.}
\begin{gather*}
    \mathbb{F} = 
    \begin{pmatrix}
        x & 0 & x & \ldots\\
        x & 0 & x & \ldots \\
        x & x & 0 & \ldots \\
        x & x & 0 & \ldots \\
        x & x & 0 & \ldots \\
        x & x & 0 & \ldots \\
        x & x & 0 & \ldots \\
    \end{pmatrix}
\end{gather*}
Первый и второй столбец --- такие факторы плохие. Второй и третий идеально. Желательно, чтобы факторы не пересекались. 
И то, что у нас есть мы можем вращать. $\mathbb{\tilde{F}} = \mathbb{F}\mathbb{W}$. $\mathbb{X} = \mathbb{V}\mathbb{F^T} = \mathbb{\tilde{V}}\mathbb{\tilde{F}}$, $\mathbb{\tilde{V}} = \mathbb{V}\mathbb{W}$. Если мы будем изменять $\mathbb{W}$ то мы будем изменять матрицу $\mathbb{F}$. 

Осталось только понять как и что нам надо улучшать? Хорошая характеристика для столбца $(1)$ и $(2)$ это стандартное отклонение.  Для первого будет равна нулю. Чем больше <<контрастность>>, тем характеристика будет больше. Метод varimax:
\begin{equation}
\sum\limits_{j = 1}^{r} \bigg[\frac{1}{p}\sum\limits_{i = 1}^{p}{(f_{ij}^2)^2} - \bigg(\frac{1}{p}\sum\limits_{i = 1}^{p}{\tilde{f}_{ij}^2}\bigg)^2\bigg] \rightarrow \max\limits_{w}
\end{equation}
\textit{Varimax}: ищем простые факторы (то есть максимальная контрастность).\textit{Quartimax}: простая интерпретация признаков (берем sd по строчкам). \textit{Equimax}: и то и то.
Важно, что после вращения $\mathbb{\tilde{V}}$ будут тоже ортогональны (столбцы). 

Для лучшей интерпретации факторов иногда разрешают неортогональные вращения: \textit{oblique, oblimin} --- косоугольные вращения. Но при таком типе вращения можно получить фактор там, где их нет. 

%%%%%%%%%%%%%%%%%%%%%%%%%%%%%%%%%%%%%%%%%%%%%%%%%%%%%%%%%%%%%%%%%%%%%%%%%%%%%%%%%%%%%%%%%%
\subsection{Билет 48. Факторная структура (корреляции исходных признаков с факторами) и факторный паттерн (коэффициенты лин. комбинации, с которыми исходные признаки выражаются через факторы) в случае ортогональных и неортогональных факторов.}
\begin{dfn}
    \textit{Factor structure} --- корреляция между исходными признаками и факторными значениями.
\end{dfn}

\begin{dfn}
    \textit{Factor pattern} --- коэффициенты линейной комбинации, как исходные признаки выражаются через факторы.
\end{dfn}
Если факторы ортогональны, то это одно и тоже. То есть это будет просто матрица $\mathbb{F}$. Но может быть неортогональны (например вращение неортогонально), то эти вещи разные.