{\color{blue} Набирала Стася}
\subsection{Билет 47. Методы нахождения факторных значений: LS и WLS (метод Бартлетта)}
  Матрица наблюдений $\mathbb X \in M_{n, p}(\R)$ в факторном анализе представляется как 
  $$\mathbb X = \mathbb V \mathbb F^\mathrm T + \eps, \mathbb V \in M_{n, r}(\R), \mathbb F \in M_{p, r}(\R), \eps \in M_{n, p}(\R).$$
  Перепишем для $\mathbb Y = \mathbb X^\mathrm T, \mathbb W = \mathbb V^\mathrm T$ (матрица $\mathbb Y = \left[Y_1:\cdots:Y_n\right]$ содержит наблюдения в столбцах):
  $$\mathbb Y = \left(\mathbb V \mathbb F^\mathrm T+ \eps\right)^\mathrm T = \mathbb F \mathbb V ^\mathrm T + \eps^\mathrm T \implies \forall i\in 1\mathbin{:}n\; Y_i = \mathbb F W_i + \eps_i.$$
  \begin{equation}\label{eq::factor_scores_regression}
    \begin{pmatrix}x_{i1} \\ \vdots \\ x_{ip}\end{pmatrix} = \mathbb F \begin{pmatrix}w_{i1} \\ \vdots \\ w_{ir}\end{pmatrix} + \begin{pmatrix}\eps_{i1} \\ \vdots \\ \eps_{ip}\end{pmatrix}
  \end{equation}
  
  \subsubsection{Ordinary Least Squares}
    При наличии уже оцененной матрицы факторных нагрузок $\mathbb F$ выражение \ref{eq::factor_scores_regression} может быть рассмотрено как задача линейной регрессии $W_i = (V^\mathrm T)_i$ на $\mathbb F$. Тогда для каждого наблюдения $Y_i$ можно построить соответствующие значения факторов
    $$\hat{W_i} = \left(\mathbb F^\mathrm T \mathbb F\right)^{-1}\mathbb F^\mathrm T Y_i.$$
    Модель линейной регрессии работает в предположении о гомоскедастичности (одинаковой вариантивности) данных, т.е. что $\eps_{i1}, \ldots, \eps_{ip}$ --- i.i.d. Модель фактороного анализа делает более слабое предположение: $\left(\eps_{i1}, \ldots, \eps_{ip}\right) \sim N(0, \diag{\left(\sigma_1^2, \ldots, \sigma_p^2\right)})$, т.е. вариативность (степень уникальности в контексте факторного анализа) у разных наблюдений разная.
  
  \subsubsection{Weighted Least Squares (метод Бартлетта)}\footnote{Я не смогла найти публикации, где бы в явном виде был изложен именно этот <<самый простой>> вариант}
    Приведём наши данные к виду, в котором они бы удовлетворяли модели линейной регрессии. Так, если $\eps_i\sim N(0, \Psi), \Psi = \cov{\eps_i} = \diag{\left(\sigma_1^2, \ldots, \sigma_p^2\right)}, \Psi^{-1/2} = \diag{\left(\frac{1}{\sigma_1}, \ldots, \frac{1}{\sigma_p}\right)}$, то 
    $$\cov{\left(\Psi^{-1/2}\eps_i\right)} = \Psi^{-1/2}\Psi\Psi^{-1/2} = \I_{p\times p} \implies \Psi^{-1/2}\eps_i \sim N(0, \I_{p\times p})$$
    Перепишем уравнение модели ФА так, чтобы уникальности в нём были вида $\Psi^{-1/2}\eps_i$:
    \begin{equation}
    \Psi^{-1/2} Y_i = \Psi^{-1/2} \mathbb F W_i + \Psi^{-1/2}\eps_i \iaoi \Psi^{-1/2} Y_i = \left(\mathbb F^\mathrm T \Psi^{-1/2}\right)^\mathrm T W_i + \Psi^{-1/2}\eps_i,
    \end{equation}
    тогда соответствующее выражение для оценки $W_i$ по МНК имеет вид
    $$\hat{W_i} = \left(\mathbb F^\mathrm T \Psi^{-1/2} \Psi^{-1/2} \mathbb F\right)^{-1}\mathbb F^\mathrm T \Psi^{-1/2}\Psi^{-1/2} Y_i = \left(\mathbb F^\mathrm T \Psi^{-1} \mathbb F\right)^{-1} \mathbb F^\mathrm T \Psi^{-1} Y_i.$$
