\subsection{Билет 29. Как интерпретировать скалярное произведение строк в матрице факторных нагрузок в АГК?}
Смотрим на билет 28. Из него мы знаем, как интерпретировать скалярное произведение строки с самой собой — это либо норма $X_i$, если анализ ведется по ковариационной матрице, либо 1, если анализ ведется по корреляционной матрице. Естественно предполжить, что примерно так же интерпретируется и скалярное произведение строк в матрице: в самом деле, пусть как и до этого $\mathbb{F} = [F_1:\ldots:F_d] = \{f_{ij}\} \in \mathbb{R}^{p\times d}$, $f_{ij} = \langle\,X_i, V_j\,\rangle$ — матрица факторных нагрузок. Тогда $i$-ая строчка $\mathbf{F}$ интерпретируется как координаты $i$-ого признака в ортонормированном базисе $V_1, …, V_d$. Отсюда,  за счет ОНБ, без всяких формул получаем, что скалярное произведение строчек $i$ и $j$  это тоже самое, что скалярное прозведение векторов $X_i$ и $X_j$. Формально:
\begin{gather*}
\sum_{j = 1}^d f_{ij}f_{kj} = \sum_{j = 1}^d \Inner{X_i}{V_j}\Inner{X_k}{V_j} =   \left(\sum_{j = 1}^d \Inner{X_i}{V_j} \right)  \left(\sum_{j = 1}^d \Inner{X_k}{V_j}\right) = 
\begin{cases}
 \Inner{X_i}{X_k}, \text{если считать по ковариационной матрице},\\ 
 \rho(X_i, X_k), \text{если считать по корреляционной матрице}.
\end{cases}
\end{gather*}

Признаки у нас центрированы, поэтому $<X_i, X_j> = cov(X_i, X_j)$\footnote{Рекомендуется убедиться, что вы понимаете, что подразумевается под скалярным произведение}. 

\subsection{Билет 30.Как нарисовать исходные орты в плоскости двух первых главных компонент?}  
Матрица $\mathbb{U}=[U_1,…, U_p]$ --- ортогональная матрица, составленная из собственных векторов ковариационной/корреляционной матрицы. Вспоминаем линейную алгебру или один из множества предыдущих вопросов и понимаем, что
столбец — координаты вектора $U_i$ в исходном базисе. Следовательно (матрица то ортогональная), строчка $U_i$ — координаты старого базиса в новом (новый = составленный из $U_1, …, U_p$). 
Поэтому $U_{i1}, U_{i2}$ — координаты $i$-ого орта в плоскости первых двух главных компонент. 

\subsection{Билет 31.Зачем и когда первые две координаты факторных нагрузок рисуются в единичном круге?}
Если АГК строился по корреляционной матрице, то 
\begin{equation}
\sum\limits_{j = 1}^{d} f^2_{ij} = 1, 
\end{equation}
где соответственно $d = \rk \mathbb{Y}$, $f_{ij}$ --- факторные нагрузки.

Таким образом, $f^2_{i1} + f^2_{i2}$ (эта сумма, естественно, меньше 1) показывает, насколько хорошо первые две компоненты отражают $i$-ый признак.
Поэтому в таком случае на единичной окружности отображается вектор, выходящий из нуля, с концом в $(f_{i1};f_{i2})$ и длина этого вектора показывает, насколько хорошо $i$-ый признак описывается в плоскости первых двух главных компонент.

\subsection{Билет 32.Чему равна норма i-го вектора из главных компонент?}
$i$-я главная компонента $j$-го индивида --- это коэффициент при $i$-м базисном векторе ($U_i$) в разложении $j$-го индивида по базису из главных направлений, $\Inner{Y_j}{U_i}$. Вектор $i$-х главных компонент --- вектор, составленный из $i$-х главных компонент всех наблюдений $Z_i = \mathbb X U_i = \sqrt{\lambda_i} V_i$ (см. \ref{q12}).

Тогда $\left\Vert Z_i \right\Vert = \left\Vert \sqrt{\lambda_i} V_i\right\vert = \sqrt{\lambda_i} \left\Vert V_i\right\Vert$, при этом по теореме о сингулярном разложении (см. \ref{thm::SVD}) $\left\lbrace V_j\right\rbrace_{j=1}^d$ -- ОН-базис, то есть $\left\Vert V_i\right\Vert = 1$. Следовательно, $$\left\Vert Z_i \right\Vert = \sqrt{\lambda_i}.$$
% Главные компоненты — координаты индивидов в базисе из главных направлений. Вспомним, что $Y = \sum\limits_{i=1}^d \sqrt{\lambda_i} u_i \Tr{v_i} = \sum u_i \Tr{z_i}$, где $z_i = \sqrt{\lambda_i}\Tr{v_i}$.
% Таким образом, $||z_i||^2 = \sum\limits_{j=1}^{n} (v_i)_j = \lambda_i \sum \limits_{i=1}^{n}|v_i|^2$. $v$ — ортонормированный вектор, поэтому сумма превращается в 1 и норма равна корню из $i$-го собственного числа матрицы $\mathbb{Y}$.

\subsection{Билет 33. Как формализовать веса для признаков и для индивидов в АГК?}
Иногда мы хотим, чтобы некоторые индивиды давали вклад больше, чем другие. Для этого нужно каждому индивиду придать определенный вес (чем больше вес, тес больше вклад индивида). 
Если хотим придать каждому индивиду вес, то в разложении Шмидта вводим меру $\mu_2$: $\mu_2({i})=\omega_i$. В итоге получаем по прежнему биортогональное разложение, но с весами:
$\Inner{V_i}{V_j} =\frac{1}{\sum\omega_k}\sum\limits_{k=1}^{n} \omega_k (V_i)_k(V_j)_k$. 
Веса на признаках — это масштаб (когда мы осознанно придаем больший вес индивиду, у которого больше разброс).

\subsection{Билет 34. Какова модель в факторном анализе?}
Модель факторного анализа
\begin{equation*}
\xi = \mathbb{F}\eta + \eps, 
\end{equation*}
где $\xi$ --- случайный вектор размерности $p$,
$\mathbb{F}$ --- матрица размерности $p \times r$, 
$\eta$ ---случайный вектор размерности $r$, $\eps$ --- случайный вектор размерности $p$. 

При этом $\cov \xi = \Sigma, \cov \eta = \mathbb{I}, \cov \eps = \diag(\sigma^2_1, \ldots, \sigma^2_p) = \Psi$.

Можно переписать все в виде:
\begin{equation*}
\Sigma = \mathbb{F}\Tr{\mathbb{F}} + \Psi.
\end{equation*}
\begin{note}
$F_i$ не может иметь вид $(0, \ldots, a, \ldots, 0)^{\mathrm T} \ne  0$, то есть не может быть факторов, уникальных для признаков!
\end{note}

Стандартно факторный анализ делается по стандартизованным признакам, то есть $\cov \xi = \cor \xi$.
$\mathbb{F}$ называется факторными нагрузками, $\eta$ -- факторными значениями.

Общностью будем называть $\sum\limits_{j = 1}^{r} f^2_{ij} = 1 - \sigma^2_i$.

\subsection{Билет 35. Что делает АГК в модели факторного анализа при равных общностях?}
Перепишем модель факторного анализа на выборочном языке.
\begin{equation*}
\mathbb{X} = \mathbb{V}\mathbb{F}^{\mathrm T} + \eps.
\end{equation*}
$\mathbb{S}$ --- выборочная ковариационная матрица. 
Хотим 
\begin{equation*}
|| \mathbb{S} - (\mathbb{F}\mathbb{F}^T + \Psi)||^2_{F} \to \min\limits_{\mathbb{F}, \Psi}
\end{equation*}
Обозначим $\tilde{\mathbb{S}} = \mathbb{F}\mathbb{F}^T + \Psi$.
Оказывается, что эта задача эквивалентна следующей:
\begin{equation*}
\begin{cases}
\sum\limits_{i \ne j}(s_{ij} - \sum\limits_{k = 1}^{r} f_{ik}f_{jk})^2 \to \min \\
(\mathbb{F}\mathbb{F}^{\mathrm T})_{ii} \le 1) 
\end{cases}
\end{equation*}
То есть, минимизации по всем элементам, кроме диагональных.
Данный метод поиска факторов называется MINRES.
Известно, что АГК эквивалентно задаче
\begin{equation}
||\mathbb{S} - \tilde{\mathbb{S}}|| \to \min
\end{equation}
Поэтому, если общность одинаковая, то АГК и MINRES решают одну и ту же задачу.