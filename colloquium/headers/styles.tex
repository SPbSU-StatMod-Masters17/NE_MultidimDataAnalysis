% \defaultfontfeatures{Ligatures={TeX},Renderer=Basic}  %% свойства шрифтов по умолчанию
% \setmainfont[Ligatures={TeX,Historic}]{Times New Roman} %% задаёт основной шрифт документа
% \setmainfont[Ligatures={TeX}]{Times New Roman} %% задаёт основной шрифт документа
% \setsansfont{Helvetica}                    %% задаёт шрифт без засечек
% \setsansfont{CMU Serif}
% \setmonofont{American Typewriter}               %% задаёт моноширинный шрифт
% \setmathrm{xits-math.otf}
% \setmathfont{xits-math.otf}
\ifxetex
\defaultfontfeatures{Ligatures={TeX},Renderer=Basic}
\usepackage{libertine}
\setmainfont[
  Ligatures=TeX,
  Extension         = .otf,
  BoldFont          = LinLibertine_RZ,
  ItalicFont        = LinLibertine_RI,
  BoldItalicFont    = LinLibertine_RZI,
]{LinLibertine_R}
\setdefaultlanguage{russian}  %% устанавливает главный язык документа
\setotherlanguage{english}
\setmathfont{xits-math.otf}
\setmathfont[range={\mathcal,\mathbfcal},StylisticSet=1]{xits-math.otf}
%\setmathfont[range=\mathbb]{xits-mathbold.otf}   
%\setmathfont{latinmodern-math.otf}
%\setmathfont{xits-math.otf}
\else
\fi

\usepackage[left=2cm,right=2cm,top=2cm,bottom=2cm]{geometry}
\parindent=1cm
