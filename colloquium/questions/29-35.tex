\subsection{Билет 29. Как интерпретировать скалярное произведение строк в матрице факторных нагрузок в АГК?}
\begin{equation*}
\langle f_{i,j},f_{k,j} \rangle =
\langle \langle X_i,V_j\rangle,\langle X_k,X_j \rangle \rangle = 
\left\langle \frac{V^{\T} X_i}{\Vert X_i\Vert}, \frac{V^{\T} X_k}{\Vert X_k \Vert}\right\rangle =
\left\langle \frac{X_i^{\T} V V^{\T}X_k } {\Vert X_i\Vert \Vert X_k \Vert}\right\rangle = 
\frac{X_i^{\T} X_k}{\Vert X_i\Vert \Vert X_k \Vert}=cos^2(X_i,X_k) .
\end{equation*}
Если признаки нормированы и центрированы, то $cos^2(X_i,X_k)=cov(X_i,X_k)$.

\subsection{Билет 30.Как нарисовать исходные орты в плоскости двух первых главных компонент?}  
Матрица $\mathbb{U}$ --- ортогональная матрица, столбцы --- координаты исходных признаков в новом базисе $\Rightarrow (U_{i1},U_{i2})$ --- координаты i-го орта в плоскости первых двух главных компонент, поэтому в плоскости первых двух главных компонент исходные орты --- это вектора с началом в полюсе и концом в $(U_{i1},U_{i2})$. 

\subsection{Билет 31.Зачем и когда первые две координаты факторных нагрузок рисуются в единичном круге?}
Если АГК строился по корреляционной матрице, то 
\begin{equation}
\sum\limits_{j = 1}^{d} f^2_{ij} = 1, 
\end{equation}
где соответственно $d = \rk \mathbb{Y}$, $f_{ij}$ --- факторные нагрузки.

Таким образом, $f^2_{i1} + f^2_{i2}$ (эта сумма, естественно, меньше 1) показывает, насколько хорошо первые две компоненты отражают $i$-ый признак.
Поэтому в таком случае на единичной окружности отображается вектор, выходящий из нуля, с концом в $(f_{i1};f_{i2})$ и длина этого вектора показывает, насколько хорошо $i$-ый признак описывается в плоскости первых двух главных компонент.

\subsection{Билет 32.Чему равна норма i-го вектора из главных компонент?}
Она равна корню из $i$-го собственного числа матрицы $\mathbb{Y}$.

\subsection{Билет 33. Как формализовать веса для признаков и для индивидов в АГК?}
Иногда мы хотим, чтобы некоторые индивиды давали вклад больше, чем другие. Для этого нужно каждому индивиду придать определенный вес (чем больше вес, тес больше вклад индивида). 
Если хотим придать каждому индивиду вес, то в разложении Шмидта вводим меру $\mu_2$: $\mu_2({i})=\omega_i$. В итоге получаем по прежнему биортогональное разложение, но с весами:$\langle V_i,V_j \rangle =\sum_{k=1} ^n \omega_k$. 
Веса на признаках —- это масштаб (когда мы осознанно придаем больший вес индивиду, у которого больше разброс).

\subsection{Билет 34. Какова модель в факторном анализе?}
Модель факторного анализа
\begin{equation*}
\xi = \mathbb{F}\eta + \eps, 
\end{equation*}
где $\xi$ --- случайный вектор размерности $p$,
$\mathbb{F}$ --- матрица размерности $p \times r$, 
$\eta$ ---случайный вектор размерности $r$, $\eps$ --- случайный вектор размерности $p$. 

При этом $\cov \xi = \Sigma, \cov \eta = \mathbb{I}, \cov \eps = \diag(\sigma^2_1, \ldots, \sigma^2_p) = \Psi$.

Можно переписать все в виде:
\begin{equation*}
\Sigma = \mathbb{F}\mathbb{F}^{\T} + \Psi.
\end{equation*}
\begin{note}
$F_i$ не может иметь вид $(0, \ldots, a, \ldots, 0)^{\T} \ne  0$, то есть не может быть факторов, уникальных для признаков!
\end{note}

Стандартно факторный анализ делается по стандартизованным признакам, то есть $\cov \xi = \cor \xi$.
$\mathbb{F}$ называется факторными нагрузками, $\eta$ -- факторными значениями.

Общностью будем называть $\sum\limits_{j = 1}^{r} f^2_{ij} = 1 - \sigma^2_i$.

\subsection{Билет 35. Что делает АГК в модели факторного анализа при равных общностях?}
Перепишем модель факторного анализа на выборочном языке.
\begin{equation*}
\mathbb{X} = \mathbb{V}\mathbb{F}^{\T} + \eps.
\end{equation*}
$\mathbb{S}$ --- выборочная ковариационная матрица. 
Хотим 
\begin{equation*}
|| \mathbb{S} - (\mathbb{F}\mathbb{F}^T + \Psi)||^2_{F} \to \min\limits_{\mathbb{F}, \Psi}
\end{equation*}
Обозначим $\tilde{\mathbb{S}} = \mathbb{F}\mathbb{F}^T + \Psi$.
Оказывается, что эта задача эквивалентна следующей:
\begin{equation*}
\begin{cases}
\sum\limits_{i \ne j}(s_{ij} - \sum\limits_{k = 1}^{r} f_{ik}f_{jk})^2 \to \min \\
(\mathbb{F}\mathbb{F}^{\T})_{ii} \le 1) 
\end{cases}
\end{equation*}
То есть, минимизации по всем элементам, кроме диагональных.
Данный метод поиска факторов называется MINRES.
Известно, что АГК эквивалентно задаче
\begin{equation}
||\mathbb{S} - \tilde{\mathbb{S}}|| \to \min
\end{equation}
Поэтому, если общность одинаковая, то АГК и MINRES решают одну и ту же задачу.