\subsection{ Билет 10. Единый подход к множественной регрессии и одномерному однофакторному дисперсионному анализу (ANOVA) }

Отправной точкой для описания обоих статистических методов является известное Дисперсионное Тождество.
В рамках ответа на вопрос сложно объяснить значимость этого равенства, однако следует иметь в виду, что корни уходят далеко за пределы статистики.
Технически теорией, лежащей в основе простой идеи проекции случайных величин на другие случайные величины, является теория Условных Математических Ожиданий (в дальнейшем, УМО).
В рамках этой теории, в частности, удается сформулировать вариант Теоремы Пифагора для случайных величин, который и называется Дисперсионным Тождеством.

\begin{thm}[Основное Дисперсионное Тождество, без доказательства]
    Пусть $\eta, \xi$ --- случайные величины, причем $\eta$, обладающая конечным вторым моментом.
    Тогда
    \begin{equation}
        \label{VI}
        \mathbb D \eta = \mathbb E(\eta - \mathbb E(\eta\, | \, \xi))^2 + \mathbb E(\mathbb E(\eta\, | \, \xi) - \mathbb E \eta)^2,
    \end{equation}
    где $\mathbb E(\eta \,|\, \xi)$ --- УМО $\eta$ <<при условии>> $\xi$.
\end{thm}

{ \footnotesize
Стоит отметить несколько фактов:
\begin{enumerate}
    \item Математическое ожидание УМО есть математическое ожидание исходной случайной величины:
        \begin{equation}
            \mathbb E \mathbb E(\eta \, | \, \xi) = \mathbb E \eta.
        \end{equation}
    \item Математическое ожидание квадрата есть на самом деле ни что иное, как квадрат нормы в пространстве $\mathrm L^2$, порожденным соответствующим распределением.
    \item Как и в любой нормальной теореме Пифагора слагаемые в правой части \eqref{VI} (точнее то, что стоит под нормой) оказываются ортогональными, то есть
        \begin{equation}
            \mathbb E [ (\eta - \mathbb E(\eta\, | \, \xi)) \mathbb E(\eta\, | \, \xi) \big ] = 0.
        \end{equation}
    \item УМО $\mathbb E(\eta \, | \, \xi)$ --- есть результат ортогонального\footnote{именно это уточнено в предыдущем пункте} проецирования
        $\eta$ на подпространство, порожденное случайной величиной $\xi$.
\end{enumerate}
}

Рассмотрим теперь ANOVA, в которой используется частный случай Основного Тождества.
\paragraph{ANOVA}
Общая постановка задачи.
Пусть есть $k$ групп и $p$ признаков, мы пытаемся проверить, что группы друг от друга не отличаются.
Формально есть $k$ (одномерных) случайных величин $\eta_1, \ldots, \eta_k$ и гипотеза формулируется так:
\begin{equation}
    \label{PreANOVA}
    \mathrm H_0: \mathcal P(\eta_1) = \mathcal P(\eta_2) = \ldots \mathcal =P(\eta_k).
\end{equation}
Эту задачу можно переформулировать следующим образом.
Пусть есть дискретный (м.б. качественный) признак $\xi$, принимающий ровно $k$ значений: $A_1, A_2, \ldots, A_k$.
Рассмотрим случайный вектор $\Tr{(\eta, \xi)}$, такой что $\mathcal P(\eta \, | \, \xi = A_i) = \mathcal P(\eta_i)$ для всех $i \in 1:k$.
Тогда гипотеза \eqref{PreANOVA} переписывается в виде:
\begin{equation}
    \label{PreANOVA2}
    \mathrm H_0^*: \mathcal P(\eta \, | \, \xi = A_i) = \mathcal P(\eta \, | \, \xi = A_j) \text{ для всех $i, j$}.
\end{equation}

В ANOVA рассматривается частный случай (модель), когда $\mathcal P(\eta_i) = \mathcal N(\mu_i, \sigma^2)$.
В рамках этой модели гипотезам \eqref{PreANOVA} и \eqref{PreANOVA2} равносильны следующие две\footnote{равносильные} гипотезы:
\begin{gather}
    \label{ANOVA}
    \mathrm H_0: \mu_1 = \mu_2 = \ldots = \mu_k.\\
    \mathrm H_0^*: \mathbb E (\eta \, | \, \xi = A_1) = \ldots = \mathbb E (\eta \, | \, \xi = A_k).
\end{gather}
Заметим, что в рамках Основного Дисперсионного Тождества гипотезу можно сформулировать следующим образом:
\begin{gather}
    \label{ANOVA_VI_H}
    \mathrm H_0: \mathbb E(\mathbb E(\eta\, | \, \xi) - \mathbb E \eta)^2 = 0,
\end{gather}
это запись по большому счету и означает, что средние внутри групп не отличаются от общего среднего.

Выборка задается следующим образом: имеется $n_i$ индивидов из $i$-той группы ($i \in 1:k$).
Обозначим $y_{ij}$ --- $j$-того индивида из $i$-той группы ($i \in 1:k$ и $j \in 1:n_i$).
Обозначим $\overline y$ --- выборочное среднее по всем индивидам, а $\overline y_i$ --- выборочное среднее индивидов $i$-той группы.
Запишем \eqref{VI} на выборочном языке:
\begin{gather}
    \label{VI_ANOVA}
    \sum_{i=1}^k \sum_{j=1}^{n_i} (y_{ij} - \overline y)^2 = \sum_{i=1}^k \sum_{j=1}^{n_i} (y_{ij} - \overline y_i)^2 + \sum_{i=1}^k n_i (\overline y_i - \overline y)^2.
\end{gather}
Убедитесь, что понимаете (хотя бы интуитивно), откуда это получается!

Далее факты из курса статистики:
\begin{gather*}
    \mathrm {SS Total} = \mathrm {SST} \overset{\mathrm{def}}{=} \widehat{\mathbb D \eta} = \sum_{i=1}^k \sum_{j=1}^{n_i} (y_{ij} - \overline y)^2 \sim \mathcal \sigma^2 \chi^2 (n - 1);\\
    \mathrm {SS Within} = \mathrm {SSW} \overset{\mathrm{def}}{=} \widehat{\mathbb E(\eta - \mathbb E(\eta\, | \, \xi))^2} = \sum_{i=1}^k \sum_{j=1}^{n_i} (y_{ij} - \overline y_i)^2\sim \sigma^2 \mathcal \chi^2 (n - k);\\
    \mathrm {SS Between} = \mathrm {SSB} \overset{\mathrm{def}}{=} \widehat{\mathbb E(\mathbb E(\eta\, | \, \xi) - \mathbb E \eta)^2} = \sum_{i=1}^k n_i (\overline y_i - \overline y)^2 \sim \sigma^2 \mathcal \chi^2 (k - 1).
\end{gather*}

Сама статистика имеет вид
\begin{gather*}
    F = \frac{\mathrm{SSB} / (k - 1)}{\mathrm{SSW} / (n - k)} \sim \mathrm F_{k-1,n-k}.
\end{gather*}
Убедитесь, что понимаете, что при условии знания распределений $\chi^2$, тут правда получается ФИШЕР!
Откуда берется числитель должно быть ясно, из представления гипотезы в виде \eqref{ANOVA_VI_H}, знаменатель тоже можно объяснить, но это чуть более тонкое дело.

Также в ANOVA рассматривают следующее корреляционное отношение\footnote{Это тоже термин из УМО.}:
\begin{gather}
    \label{ANOVA_cor}
    r^2 = \hat r^2(\eta \, | \, \xi) = \mathrm{\frac{SSB}{SST}} = 1 - \mathrm{\frac{SSW}{SST}} = 1 - \frac{\min_{\varphi} \widehat {\mathbb E}(\eta - \varphi(\xi))^2}{\widehat {\mathbb D} \eta}.
\end{gather}

\paragraph{Multiple Linear Regression}
Здесь мы не будем касаться вопросов построения МНК-оценок, а сразу перейдем к проверке значимости регрессии.
Уточним смысл сказанного. Во-первых, вспомним, что по определению регрессия --- УМО.
Во-вторых, какая бы модель у нас ни была, в стандартной ситуации нам нужно от этой модели только, чтобы УМО было линейным
(это будет ясно чуть далее).
Существует несколько моделей, в которых УМО --- линейно. Приведем примеры двух из них.

\begin{gather*}
    \eta = \Tr{B} \overline \xi + \varepsilon,
\end{gather*}
где $\overline \xi \in \R^k$ --- случайный вектор, $\eta \in \R$, $\eps \in \R$ --- 
случайные величины, $B \in \mathrm \R^k$ --- вектор коэффициентов,
причем $\overline \xi$, $\varepsilon$ --- независимы и $\mathbb E\varepsilon = 0$.

%На генеральном языке можно сказать\footnote{Хотя, как я понимаю, это идет несколько вразрез с исходной постановкой задачи},
%что у нас есть нормальный вектор $\xi = \Tr{(\xi_1, \ldots, \xi_k)}$ и мы строим его регрессию на некоторую другую случайную величину $\eta$.

В такой модели нетрудно посчитать, что $\mathbb E(\eta \, | \, \overline \xi) = \Tr{B} \overline \xi$,
то есть регрессия является линейной, нас это устраивает.

Примером другой модели такого типа является следующая: $\Tr{(\eta, \overline \xi)} \sim \mathcal N$ с
какими-то параметрами, которые нас сейчас не интересуют. Важно то, что в такой модели
тоже найдется вектор $B$: $\mathbb E(\eta \, | \, \overline \xi) = \Tr{B} \overline \xi$.

Гипотеза о незначимости регрессии в терминах MLR формулируется следующим образом:
\begin{gather}
    \label{MLR_H}
    \mathrm H_0: B = 0.
\end{gather}
Именно здесь используется то, что мы требуем линейности регрессии (найдется такой вектор!).

Учитывая то обстоятельноство, что для независимых $\xi_1, \xi_2$ справедливо, что $\mathbb E (\xi_1 \, | \, \xi_2) = \mathbb E\xi_1$,
гипотеза о незначимости регрессии для моделей можно свести к другой, сформулированной в рамках Основного Дисперсионного Тождества
\eqref{VI}:
\begin{gather}
    \label{MLR_VI_H}
    \mathrm H_0: \mathbb E(\mathbb E(\eta\, | \, \overline \xi) - \mathbb E \eta)^2 = 0.
\end{gather}
Несложно понять, что эта запись есть то же самое, что и \eqref{MLR_H}.

Заметим, что вторая гипотеза является гораздо более общей, в ней не нужны предположения о линейности регрессии!
Обратим внимание, что эта гипотеза полностью совпадает по своему виду с гипотезой \eqref{ANOVA_VI_H},
которая проверяется с помощью ANOVA. 

Выборка задана набором $\{y_i\}_{i=1}^n$. Обозначим $\hat y_i$ --- наилучшее\footnote{Все должны помнить в каком смысле.} по МНК предсказание $y_i$.
Запишем \eqref{VI} на выборочном языке:
\begin{gather}
    \label{VI_MLR}
    \sum_{i=1}^n (y_i - \overline y)^2 = \sum_{i=1}^n (y_{i} - \hat y_i)^2 + \sum_{i=1}^n (\hat y_i - \overline y)^2.
\end{gather}
Убедитесь, что понимаете (хотя бы интуитивно), откуда это получается!

Далее факты из курса статистики:
\begin{gather*}
    \mathrm {SS Total} = \mathrm {SST} \overset{\mathrm{def}}{=} \widehat{\mathbb D \eta} = \sum_{i=1}^n (y_i - \overline y)^2 \sim \mathcal \sigma^2 \chi^2 (n - 1);\\
    \mathrm {SS Error} = \mathrm {SSE} \overset{\mathrm{def}}{=} \widehat{\mathbb E(\eta - \mathbb E(\eta\, | \, \xi))^2} = \sum_{i=1}^n (y_{i} - \hat y_i)^2 \sim \sigma^2 \mathcal \chi^2 (n - k - 1);\\
    \mathrm {SS Regression} = \mathrm {SSR} \overset{\mathrm{def}}{=} \widehat{\mathbb E(\mathbb E(\eta\, | \, \xi) - \mathbb E \eta)^2} = \sum_{i=1}^n (\hat y_i - \overline y)^2 \sim \sigma^2 \mathcal \chi^2 (k).
\end{gather*}

Сама статистика имеет вид
\begin{gather*}
    F = \frac{\mathrm{SSR} / k}{\mathrm{SSE} / (n - k - 1)} \sim \mathrm F_{k,n-k-1}.
\end{gather*}
Убедитесь, что понимаете, что при условии знания распределений $\chi^2$, тут правда получается ФИШЕР!

Так же при регрессии рассматривают множественный коэффициент корреляции между $\eta$, $\xi$:
\begin{gather}
    \nonumber
    r^2 = \hat {\mathrm R}^2(\eta; \{\xi_1, \xi_2, \ldots, \xi_k\}) = \mathrm{\frac{SSR}{SST}} = 1 - \mathrm{\frac{SSE}{SST}} = \\ =
    \label{MLR_coef_cor}
    хЬ№О1 - \frac{\min_{a_0, a_1, \ldots, a_k} \hat {\mathbb E}(\eta - (a_0 + \sum_{i=1}^k a_i \xi_i))^2}{\widehat {\mathbb D} \eta}.
\end{gather}

\paragraph{Выводы}
Я думаю, что даже тот, кто просматривал этот текст одним глазом заметил, что я написал два раза почти одно и то же.
Теперь дадим некоторый общий (и на самом деле удобный) взгляд на то, что было раньше.

Начнем с гипотез. Заметим, что \eqref{ANOVA_VI_H} и \eqref{MLR_VI_H} вообще ничем не отличается, поэтому можно ввести следующую удобную универсальную терминологию (шляпа, как всегда обозначает оценки, смысл которых уточняется в частных случаях).
\begin{gather*}
    \mathrm {Hypothesis} \overset{\mathrm{def}}{=} \mathbb E(\mathbb E(\eta\, | \, \xi) - \mathbb E \xi)^2.\\
    \mathrm {Error} \overset{\mathrm{def}}{=} \mathbb E(\eta - \mathbb E(\eta\, | \, \xi))^2;\\
    \mathrm {SS Total} = \mathrm {SST} \overset{\mathrm{def}}{=} \widehat{\mathbb D \eta};\\
    \mathrm {SS Error} = \mathrm {SSE} \overset{\mathrm{def}}{=} \widehat{\mathbb E(\eta - \mathbb E(\eta\, | \, \xi))^2};\\
    \mathrm {SS Hypothesis} = \mathrm {SSH} \overset{\mathrm{def}}{=} \widehat{\mathbb E(\mathbb E(\eta\, | \, \xi) - \mathbb E \eta)^2}.
\end{gather*}

В этих обозначениях общая гипотеза будет иметь вид:
\begin{gather*}
    \mathrm H_0: \mathrm{Hypothesis} = 0.
\end{gather*}

А Дисперсионное Тождество на выборочном языке:
\begin{gather*}
    \mathrm{SST = SSH + SSE}.
\end{gather*}

Статистика в общем виде будет выглядеть, как
\begin{gather*}
    F = \frac{\mathrm{SSH} / \nu_H}{\mathrm{SSE} / \nu_E} \sim \mathrm F_{\nu_H, \nu_E},
\end{gather*}
где $\nu_\cdot$ --- нормировочные числа, отвечающие за степени свободы.

На будущее вводим обозначения:
\begin{gather*}
    \lambda \overset{\mathrm{def}}{=} \mathrm{\frac{SSH}{SSE}};\\
    \Lambda \overset{\mathrm{def}}{=} \frac{1}{1 + \lambda} \sim \Lambda_1 (\nu_H, \nu_E);\footnotemark
\end{gather*}
\footnotetext{Докажите это! НЭ будет это спрашивать. Удачи.}

Обобщающее \eqref{ANOVA_cor} и \eqref{MLR_coef_cor} определение:
\begin{gather*}
    r^2 = \mathrm{\frac{SSH}{SST}}. 
\end{gather*}


\subsection{Билет 11. Представление одномерного однофакторного дисперсионного анализа в виде множественной регрессии с фиктивными переменными.}

Вот есть ANOVA и есть многомерная регрессия... У них много общего. Сейчас мы покажем, что на самом деле на ANOVA можно посмотреть, как на частный случай многомерной регрессии.
Будем пользоваться обозначениями билета 10.
Для $i \in 1:{k-1}$ введем $\xi_i = \mathbb I (\xi = A_i)$ -- фиктивные переменные.

\begin{thm}
    $r^2(\eta \, | \, \xi) = \mathrm R^2 (\eta, \{\xi_1, \xi_2, \ldots, \xi_{k-1}\})$.
\end{thm}
\begin{proof}
    Распишем по определению обе части равенства.
    \begin{gather*}
        r^2(\eta \, | \, \xi) = 1 - \frac{\min_{\varphi} \mathbb E(\eta - \varphi(\xi))^2}{\mathbb D \eta};\\
        \mathrm R^2(\eta; \{\xi_1, \xi_2, \ldots, \xi_k\}) = 1 - \frac{\min_{a_0, a_1, \ldots, a_k} {\mathbb E}(\eta - (a_0 + \sum_{i=1}^k a_i \xi_i))^2}{{\mathbb D} \eta}.
    \end{gather*}
    Минимум в корреляционном отношении идет по всем измеримым функциям, каждая из которых однозначно определяется набором значений $\{\varphi(A_i)\}$.
    Пусть минимум в корреляционном отношении достигается на функции $\varphi^*$. Положим $a_i = \varphi^*(A_i)$ для $i \in 1:(k-1)$ и $a_0 = \varphi^*(A_k)$.
    От противного легко доказать, что именно на этом наборе коэффициентов достигается минимум в коэффициенте корреляции.
    В обратную сторону -- аналогично.\footnote{\color{blue} Вроде бы тут все корректно. Проверьте, кто может.}
\end{proof}

\subsection{Билет 12. Корреляционное отношение с дискретным одномерным признаком и множественный коэффициент корреляции.}

По-моему, это тот же 11ый билет.
